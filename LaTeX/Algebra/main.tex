\documentclass[12pt]{article}
\usepackage[utf8]{inputenc}
\usepackage[T1]{fontenc}
\usepackage[english]{babel}
\usepackage{amsmath, amssymb, amsthm, mathtools}
\usepackage{geometry}
\geometry{margin=1in}
\usepackage{lmodern}
\usepackage{hyperref}
\usepackage{xcolor}
\usepackage{enumitem}
\usepackage[backend=bibtex, style=numeric]{biblatex}
\addbibresource{references.bib}
\title{Algebra Cheat Sheet}
\author{Giorgina}
\date{\today}
\begin{document}
\maketitle

This cheat sheet is based on material from Coursera's Algebra I course \cite{coursera-algebra-i}.


\section*{Sets of Numbers}

\begin{tabular}{|l|l|p{8cm}|}
\hline
\textbf{Name} & \textbf{Symbol} & \textbf{Description} \\
\hline
Natural Numbers & $\mathbb{N}$ & Counting numbers: $\{1, 2, 3, \dots\}$ \\
Whole Numbers & $\mathbb{W}$ & Natural numbers plus zero: $\{0, 1, 2, 3, \dots\}$ \\
Integers & $\mathbb{Z}$ & All whole numbers and negatives: $\{\dots, -3, -2, -1, 0, 1, 2, 3, \dots\}$ \\
Rational Numbers & $\mathbb{Q}$ & Can be written as a ratio of integers \\
Irrational Numbers & --- & Cannot be written as a ratio of integers (e.g., $\pi$, $\sqrt{2}$) \\
Real Numbers & $\mathbb{R}$ & All rational and irrational numbers \\
\hline
\end{tabular}

\vspace{1em}
\textbf{Note:} Sometimes $\mathbb{N}$ includes zero, but for this course it does not.

\section*{Closure Properties of Real Numbers}

\textbf{Addition and Multiplication:} If $a, b \in \mathbb{R}$, then $(a + b) \in \mathbb{R}$ and $(a \cdot b) \in \mathbb{R}$.

\textbf{Subtraction:} If $a, b \in \mathbb{R}$, then:
\[
a - b = a + (-b) \in \mathbb{R}.
\]

\textbf{Division:} If $a, b \in \mathbb{R}$ and $b \neq 0$, then:
\[
a \div b = a \cdot \frac{1}{b} \in \mathbb{R}.
\]

\section*{Properties of Addition and Multiplication}

\begin{tabular}{|l|l|p{8cm}|}
\hline
\textbf{Property} & \textbf{Equation} & \textbf{Description} \\
\hline
Commutative (Add) & $a + b = b + a$ & Order doesn't change sum \\
Commutative (Mult) & $a \cdot b = b \cdot a$ & Order doesn't change product \\
Associative (Add) & $(a + b) + c = a + (b + c)$ & Grouping doesn't change sum \\
Associative (Mult) & $(a \cdot b) \cdot c = a \cdot (b \cdot c)$ & Grouping doesn't change product \\
Distributive & $a \cdot (b + c) = a \cdot b + a \cdot c$ & Distribution over addition \\
\hline
\end{tabular}

\section*{Order of Operations}

The order of operations is:
\begin{enumerate}[label=\arabic*.]
    \item \textbf{Parentheses/Brackets}
    \item \textbf{Exponents/Orders}
    \item \textbf{Multiplication and Division (left to right)}
    \item \textbf{Addition and Subtraction (left to right)}
\end{enumerate}
Common acronyms: PEMDAS, BODMAS, BEDMAS

\section*{Fractions}

\textbf{For $\frac{a}{b}$:}
\begin{itemize}
    \item Numerator: $a$
    \item Denominator: $b$
    \item Reciprocal: $\frac{b}{a}$
\end{itemize}

\subsection*{Multiplication}
\[
\frac{a}{b} \cdot \frac{c}{d} = \frac{a \cdot c}{b \cdot d}
\]

\subsection*{Division}
\[
\frac{a}{b} \div \frac{c}{d} = \frac{a}{b} \cdot \frac{d}{c} = \frac{a \cdot d}{b \cdot c}
\]

\subsection*{Addition and Subtraction}
\[
\frac{a}{b} + \frac{c}{d} = \frac{ad + cb}{bd}
\]
\[
\frac{a}{b} - \frac{c}{d} = \frac{ad - cb}{bd}
\]

\subsection*{Fractions and Whole Numbers}
\[
a = \frac{a}{1}
\]

\section*{Equations and Properties of Equality}

\subsection*{Definitions}

\begin{itemize}
    \item \textbf{Equation:} expression with equals sign.
    \item \textbf{Coefficient:} multiplier of a variable.
    \item \textbf{Solution:} value(s) that make equation true.
    \item \textbf{Degree:} highest power of variable.
    \item \textbf{Linear equation:} equation with degree 1.
\end{itemize}

\subsection*{Properties of Equality}

\begin{tabular}{|l|l|p{8cm}|}
\hline
\textbf{Property} & \textbf{Rule} & \textbf{Explanation} \\
\hline
Symmetric & If $a = b$, then $b = a$ & Order of equality doesn't matter \\
Transitive & If $a = b$ and $b = c$, then $a = c$ & Equal values substitute \\
Addition & If $a = b$, then $a + c = b + c$ & Adding same value preserves equality \\
Subtraction & If $a = b$, then $a - c = b - c$ & Subtracting same value preserves equality \\
Multiplication & If $a = b$, then $a \cdot c = b \cdot c$ & Multiplying preserves equality \\
Division & If $a = b$, then $a \div c = b \div c$ ($c \neq 0$) & Division preserves equality \\
\hline
\end{tabular}

\subsection*{Solving Linear Equations}

\begin{enumerate}[label=\arabic*.]
    \item Simplify both sides using order of operations.
    \item Move constants to one side.
    \item Move variables to the other side.
    \item Divide by the coefficient of the variable.
\end{enumerate}

\section*{Properties of Absolute Value}

\begin{tabular}{|l|l|p{8cm}|}
\hline
\textbf{Property} & \textbf{Rule} & \textbf{Explanation} \\
\hline
Idempotence & $\left| \left| x \right| \right| = \left| x \right|$ & Absolute value taken multiple times doesn’t change value. \\
Symmetry & $\left| -x \right| = \left| x \right|$ & Opposite sign has same absolute value. \\
Multiplicative & $\left| x \cdot y \right| = \left| x \right| \cdot \left| y \right|$ & Absolute value distributes over multiplication. \\
Preserves Division & $\left| \frac{x}{y} \right| = \frac{\left| x \right|}{\left| y \right|}$ & Absolute value distributes over division. \\
\hline
\end{tabular}

\section*{Triangle Inequality}

\textbf{Be careful:} In general:
\[
\left| x + y \right| \neq \left| x \right| + \left| y \right|.
\]

\textbf{Example:}

Let $x = 2$, $y = -1$:

\[
\left| x + y \right| = \left| 2 + (-1) \right| = \left| 1 \right| = 1
\]

\[
\left| x \right| + \left| y \right| = \left| 2 \right| + \left| -1 \right| = 2 + 1 = 3
\]

\textbf{Triangle Inequality:}
\[
\left| x + y \right| \leq \left| x \right| + \left| y \right|
\]

\section*{Solving Absolute Value Equations}

If solving $\left| x \right|$, remember that:
\[
\left| x \right| = 
\begin{cases}
x & \text{if } x \geq 0 \\
-x & \text{if } x < 0
\end{cases}
\]

\textbf{Solving process:}
\begin{enumerate}[label=\arabic*.]
    \item Write two equations: one with $x$, one with $-x$.
    \item Solve both equations.
    \item Verify solutions satisfy the original equation.
    \item Usually, expect two solutions.
\end{enumerate}

\printbibliography

\end{document}
